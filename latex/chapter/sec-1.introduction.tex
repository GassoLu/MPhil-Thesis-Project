\chapter{Introduction}\label{sec-1.introduction}

Why did exchange rate fluctuations not result in price changes of the same magnitude? This is one of the core questions among a set of "exchange rate disconnect" puzzles (Obstfeld and Rogoff, 2000\cite{obstfeld2000}). As price signals in the international trade market, exchange rates appear less informative for firms than expected. Exchange rate pass-through (ERPT), which describes the elasticity of local price changes to exchange rate fluctuations, varies widely across countries, industries, and time. Existing studies generated widely varying estimates of this exchange rate pass-through. Understanding this pricing elasticity is of particular interest to researchers both in the fields of international trade and open macroeconomics to demystify the exchange rate disconnect.

A large body of theoretical and empirical work explores the mechanism of incomplete exchange rate pass-through in prices. Most micro explanations and empirical evidence based on disaggregated data for incomplete exchange rate pass-through focus on the exporter sides. Exporters' productivity (Berman, Martin, and Mayer, 2012\cite{bmm2012}; Li, Ma, and Xu, 2015\cite{lmx2015}) and product quality (Chen and Juvenal, 2016\cite{chen2016}; Auer, Chaney, and Sauré, 2018\cite{auer2018}) , as well as their imported inputs (Amiti, Itskhoki, and Konings, 2014\cite{aik2014}; Wang and Yu, 2021\cite{wang-yu2021}) and market shares (Auer and Schoenle, 2016\cite{auer2016}; Devereux, Dong, and Tomlin, 2017\cite{devereux2017}), will all affect the export exchange rate pass-through. Yet the direct role of importers in determining exchange rate pass-through remains a novel field to study. 

Financial constraints, discussed in another strand of literature, also demonstrate influence on firms’ response in price-setting decisions to exchange rate fluctuations. Strasser (2013)\cite{strasser2013} first finds that financially constrained exporters adjust prices in the destination market more sharply when facing exchange rate shocks, implying a more complete exchange rate pass-through. Firms with tighter financial constraints tend to set higher prices due to higher external financing premiums and the resulting higher marginal costs and face higher price elasticity of demand. Thus, with endogenously determined markups, exchange rate depreciation (appreciation) allows firms to increase (decrease) markups, but credit-constrained firms do so only to a limited extent because they have less space to adjust their profit margins. However, it remains an open question whether importers under financial constraints will behave differently in price negotiation during exchange rate shocks. Therefore, credit constraints provide an innovative entry point for us to study the import exchange rate pass-through.

In this paper, we focus on the role of importers in the determination of exchange rate pass-through and connect it with the importers' financial constraints. This paper tends to fill a gap in the literature by linking both sides of the trade relationship and provides a novel perspective to study the nature of exchange rate disconnect for emerging markets, where firms are more vulnerable to credit constraints due to immature financial markets. In contrast to the conventional framework of exchange rate pass-through where importers are mostly price takers, we contribute to the trade literature by identifying the importers' implicit sourcing power by comparing the heterogeneous capacity to absorb exchange rate shocks. Throughout the paper, we will compare firm-level import exchange rate pass-through with the export pass-through to reflect the similarities and differences between the two. Going a step further, if a country's export and import exchange rate pass-through patterns differ significantly, this may even affect terms of trade as well as current account imbalances.

We estimate the import exchange rate pass-through as the price elasticity of import prices concerning real exchange rates using Chinese firm-level data. Specifically, we merge the Chinese Industrial Enterprises datasets with the China customs data and adopted fixed effects panel regressions with first-order differences to capture the changes in product prices and real exchange rates. The average import exchange rate pass-through is between 35\%-40\%, which is more incomplete compared to the over 95\% export pass-through. Second, we identify the effects of credit constraints on importers' exchange rate pass-through. I use both US measures of sectors’ financial vulnerability (Manova, Wei, and Zhang, 2015\cite{manova-wei-zhang2015}) and Chinese measures of credit needs (Fan, Lai, and Li, 2015\cite{fan-li-yeaple2015}). In our baseline results, import prices for firms in sectors with higher financial constraints are more sensitive to exchange rate shocks. Third, we study some potential channels under which credit constraints may affect the import pass-through. We estimate the firm-level markup and total product productivities (TFP) following De Loecker and Warzynski (2012)\cite{dlw2012} and calculate the proxy for an importer's sourcing base. We find that although firm heterogeneity does affect exchange rate pass-through, credit constraints still affect import pass-through significantly even if we control both those firm characteristics. 

In robustness checks, we further use alternative measures of credit constraints and alternative subsamples. For credit constraints, we compare sectoral-level measures calculated from China data with those from US data. For subsamples, we divide our subjects into two subsets: two-way traders (simultaneous import and export) and one-way traders (either purely import or purely export). We check our results using only the two-way traders who account for the vast majority of the number of traders and the vast majority of China's trade volume. These results are all significant and robust.

We show that importers‘ credit constraints do influence price-setting patterns in international trade. We provide evidence for three key findings: (1) average import exchange rate pass-through levels in China are significantly less complete than the export ones; (2) financial constraints will increase both of them to be more complete, and (3) importers who import a certain product from more sources have a less complete pass-through. In other words, financially constrained importers will absorb more price fluctuations caused by exchange rate changes, while financially constrained exporters pass through more exchange rate changes to prices, both compared to those unconstrained firms. This reflects that binding financial constraints will lead to not only narrow margins to adopt pricing-to-market strategies for the sellers but also limited sourcing power for the buyers. Importers with a wider sourcing base could get access to alternative options and avoid bargaining disadvantages in exchange rate fluctuations to some extent.

The remainder of the paper is organized as follows. Section 2 presents a more detailed literature review. Section 3 describes the data we used. Section 4 introduces our empirical strategy and measures of key variables. Section 5 shows the main empirical results about import exchange rate pass-through and credit constraints. Section 6 provides robustness checks and some discussion. Section 7 concludes.

\newpage
